\documentclass[hyperref=unicode,presentation,10pt]{beamer}

\usepackage[absolute,overlay]{textpos}
\usepackage{array}
\usepackage{graphicx}
\usepackage{adjustbox}
\usepackage[version=4]{mhchem}
\usepackage{chemfig}
\usepackage{caption}

%dělení slov
\usepackage{ragged2e}
\let\raggedright=\RaggedRight
%konec dělení slov

\addtobeamertemplate{frametitle}{
	\let\insertframetitle\insertsectionhead}{}
\addtobeamertemplate{frametitle}{
	\let\insertframesubtitle\insertsubsectionhead}{}

\makeatletter
\CheckCommand*\beamer@checkframetitle{\@ifnextchar\bgroup\beamer@inlineframetitle{}}
\renewcommand*\beamer@checkframetitle{\global\let\beamer@frametitle\relax\@ifnextchar\bgroup\beamer@inlineframetitle{}}
\makeatother
\setbeamercolor{section in toc}{fg=red}
\setbeamertemplate{section in toc shaded}[default][100]

\usepackage{fontspec}
\usepackage{unicode-math}

\usepackage{polyglossia}
\setdefaultlanguage{czech}

\def\uv#1{„#1“}

\mode<presentation>{\usetheme{default}}
\usecolortheme{crane}

\setbeamertemplate{footline}[frame number]

\title[Crisis]
{C2062 -- Anorganická chemie II}

\subtitle{(Uhlík, křemík,) germanium, cín, olovo a flerovium}
\author{Zdeněk Moravec, hugo@chemi.muni.cz \\ \adjincludegraphics[height=60mm]{img/IUPAC_PSP.jpg}}
\date{}

\title[Crisis]
{C2062 -- Anorganická chemie II}

\subtitle{(Dusík, fosfor,) arsen, antimon, bismut a moscovium}
\author{Zdeněk Moravec, hugo@chemi.muni.cz \\ \adjincludegraphics[height=60mm]{img/IUPAC_PSP.jpg}}
\date{}

\begin{document}

\begin{frame}
	\titlepage
\end{frame}

\section{Dusík}
%Haber-Bosch; NI3; hypergolické látky
\frame{
	\frametitle{}
	\begin{columns}
		\begin{column}{0.6\textwidth}
			\vfill
			\begin{itemize}
				\item Značka N, protonové číslo 7.
				\item Za laboratorní teploty je plynný, kapalný dusík se běžně využívá jako ochranná atmosféra a jako chladivo.
				\item Molekula \ce{N2} obsahuje trojnou vazbu \ce{N\bond{3}N}, která ji dává vysokou stabilitu a inertnost.
				\item Tvoří největší podíl v Zemské atmosféře.
				\item Koloběh dusíku na Zemi popisuje fixaci atmosférického dusíku a jeho opětovné uvolňování biologickými systémy.
				\item Jednou z nejdůležitějších sloučenin dusíku je amoniak, \ce{NH3}.
			\end{itemize}
			\vfill
		\end{column}
		\begin{column}{0.5\textwidth}
			\begin{figure}
				\adjincludegraphics[width=\textwidth]{img/Liquidnitrogen.jpg}
				\caption*{Kapalný dusík.\footnote[frame]{Zdroj: \href{https://commons.wikimedia.org/wiki/File:Liquidnitrogen.jpg}{Cory Doctorow/Commons}}}
			\end{figure}
		\end{column}
	\end{columns}
}

\frame{
	\frametitle{}
	\begin{figure}
		\adjincludegraphics[width=.8\textwidth]{img/Nitrogen_cycle_cs.png}
		\caption*{Koloběh dusíku.\footnote[frame]{Zdroj: \href{https://commons.wikimedia.org/wiki/File:Nitrogen_cycle_cs.svg}{Nojhan/Commons}}}
	\end{figure}
}

\frame{
	\frametitle{}
	\begin{itemize}
		\item \textbf{Amoniak}, \ce{NH3}, bezbarvý, toxický plyn.
		\item Díky vysoké polaritě se dobře rozpouští ve vodě (31~\% při 25~$^\circ$C).
		\item Podobně jako voda má schopnost autoionizace:
		\item \ce{NH3 + NH3 <=> NH$^+_4$ + NH$^-_2$}
		\item Stupnice pH v kapalném amoniaku má rozsah 0--30 (K = 10$^{-30}$).
		\item Roztoky alkalických kovů v kapalném amoniaku poskytují tzv. solvatovaný elektron.\footnote{\href{https://en.wikipedia.org/wiki/Solvated_electron}{Solvated electron}}
		\item Světová produkce amoniaku se blíží množství 200 miliónů tun.
		\item Využívá se při výrobě hnojiv a dalších sloučenin dusíku, jako rozpouštědlo, dříve se využíval i v ledničkách.
		\item Přímá syntéza z prvků je v průmyslovém měřítku neproveditelná.
		\item V roce 1913 byla vyvinuta Haberova--Boschova metoda výroby amoniaku s využitím heterogenních katalyzátorů na bázi oxidů železa.
	\end{itemize}
}

\frame{
	\frametitle{}
	\begin{figure}
		\adjincludegraphics[width=.82\textwidth]{img/Potential_energy_diagram_for_ammonia_synthesis.png}
		\caption*{Energetický profil syntézy amoniaku.\footnote[frame]{Zdroj: \href{https://commons.wikimedia.org/wiki/File:Potential_energy_diagram_for_ammonia_synthesis.svg}{Imalipusram/Commons}}}
	\end{figure}
}

\frame{
	\frametitle{}
	\begin{figure}
		\adjincludegraphics[width=1.1\textwidth]{img/Haber-Bosch.png}
		\caption*{Haberova-Boschova syntéza amoniaku.\footnote[frame]{Zdroj: \href{https://commons.wikimedia.org/wiki/File:Haber-Bosch-En.svg}{Francis E Williams/Commons}}}
	\end{figure}
}

\frame{
	\frametitle{}
	\begin{itemize}
		\item \textbf{Jododusík}, \ce{NI3}, je černozelená pevná látka.
		\item V suchém stavu ho lze velmi snadno přivést k explozi.
		\item Připravuje se reakcí roztoku amoniaku s jodem kdy vzniká adukt \ce{NI3.NH3}.
		\item V roce 1990 byl připraven čistý \ce{NI3} reakcí BN s \ce{IF}:\footnote{\href{https://doi.org/10.1002/anie.199006771}{Nitrogen Triiodide}}
		\item \ce{BN + 3 IF ->[CFCl3][$-30 ^\circ$C] NI3 + BF3}
	\end{itemize}
	\begin{figure}
		\adjincludegraphics[width=.9\textwidth]{img/NH3-NI3-chain-from-xtal-3D-bs-20.png}
		\caption*{Struktura aduktu \ce{NI3.NH3}.\footnote[frame]{Zdroj: \href{https://commons.wikimedia.org/wiki/File:NH3·NI3-chain-from-xtal-3D-bs-20.png}{Ben Mills/Commons}}}
	\end{figure}
}

\section{Fosfor}
\frame{
	\frametitle{}
	\begin{itemize}
		\item Značka P, protonové číslo 15.
		\item Pevná látka, vytváří řadu allotropních modifikací.
		\item Důležitý biogenní prvek, vyskytuje se např. v kostech, zubech, ATP (ADP, AMP).
		\item Fosforečnany jsou součástí hnojiv.
	\end{itemize}
	\begin{figure}
		\adjincludegraphics[width=.9\textwidth]{img/PhosphComby.jpg}
		\caption*{Bílý, červený a fialový fosfor.\footnote[frame]{Zdroj: \href{https://commons.wikimedia.org/wiki/File:PhosphComby.jpg}{Commons}}}
	\end{figure}
	\vfill
}

\frame{
	\frametitle{}
	\begin{itemize}
		\item \textit{Bílý fosfor} -- skládá se z jednotek \ce{P4}, je velmi reaktivní, na vzduchu je samozápalný.
		\item \textit{Červený fosfor} -- amorfní modifikace, má polymerní strukturu. Připravuje se zahříváním bílého fosforu v anaerobních podmínkách. Je výrazně méně reaktivní než bílý fosfor.
		\item \textit{Černý fosfor} -- má kovový lesk, vlastnostmi se blíží kovům. Vzniká zahříváním červeného fosforu pod tlakem.
		\item \textit{Fialový (Hittorfův) fosfor} -- je tvořen řetězci fosforu propojenými do polymerní sítě. Poprvé byl připraven krystalizací fosforu z taveniny olova.
	\end{itemize}
	\begin{figure}
		\adjincludegraphics[height=.2\textheight]{img/Violet-phosphorus-chain-from-xtal-3D-balls.png}
		\caption*{Krystalová struktura fialového fosforu.\footnote[frame]{Zdroj: \href{https://commons.wikimedia.org/wiki/File:Violet-phosphorus-chain-from-xtal-3D-balls.png}{Benjah-bmm27/Commons}}}
	\end{figure}
	\vfill
}

\frame{
	\frametitle{}
	\begin{itemize}
		\item \textit{Oxid fosforečný} -- bílý, hygroskopický prášek. Využívá se jako sušidlo i důležitá výchozí látka pro další sloučeniny fosforu.
		\item \ce{P4O10 + 6 H2O -> 4 H3PO4}
		\item Získává se oxidací bílého fosforu na vzduchu.
		\item \ce{P4 + 5 O2 -> P4O10}
		\item Má strukturu adamantanu, šest kyslíků je endocyklických a čtyři exocyklické.
		\item Známe pět oxidů v rozmezí \ce{P4O6} až \ce{P4O10}, které se liší počtem exocyklických kyslíků.
	\end{itemize}
	\begin{figure}
		\adjincludegraphics[width=\textwidth]{img/Structures_of_phosphorus_oxides.png}
		\caption*{Oxidy fosforu v oxidačním stavu V.}
	\end{figure}
	\vfill
}

\frame{
	\frametitle{}
	\begin{itemize}
		\item \textit{Oxokyseliny fosforu}:
		\item Vždy obsahují skupinu \ce{P=O}.
		\item Vždy obsahují alespoň jednu skupinu \ce{P-OH}, která je ionizovatelná.
		\item Vazby \ce{P-H} nejsou ionizovatelné.
		\item Řetězení probíhá přes vazby \ce{P-O-P} nebo \ce{P-P}.
		\item Pro přípravu sloučenin fosforu jsou důležité jak kyseliny, tak jejich estery.
	\end{itemize}
	\begin{figure}
		\adjincludegraphics[width=\textwidth]{img/P-kyseliny.png}
	\end{figure}
	\vfill
}

\frame{
	\frametitle{}
	\begin{itemize}
		\item \textit{Fosfazeny} -- nenasycené sloučeniny fosforu a dusíku.
		\item Obsahují zpravidla \ce{P^V} a vazbu \ce{P=N}.
		\item Monofosfazeny připravil A. V. Kirsanovov v roce 1962:
		\item \ce{Ph3PCl2 + PhNH2 -> Ph3P=NPh + 2 HCl}
		\item Polyfosfazeny můžeme připravit reakcí \ce{PCl5} s amoniakem:
		\item \ce{n PCl5 + n NH4Cl ->[120-150 $^\circ$C] (NPCl2)_n + 4n HCl}
	\end{itemize}
	\begin{figure}
		\adjincludegraphics[width=\textwidth]{img/Phosphornitrilchloride.png}
	\end{figure}
}

\section{Úvod -- arsen, antimon, bismut a moscovium}
\frame{
	\frametitle{}
	\vfill
	\begin{tabular}{|c|l|l|l|}
	\hline
	 & \textit{Arsen} & \textit{Antimon} & \textit{Bismut} \\\hline
	 El. konfigurace & 3d$^{10}$ 4s$^{2}$ 4p$^{3}$ & 4d$^{10}$ 5s$^{2}$ 5p$^{3}$ & 4f$^{14}$ 5d$^{10}$ 6s$^{2}$ 6p$^{3}$ \\\hline
	 Teplota tání [$^\circ$C] & 614 & 631 & 272 \\\hline
	 Teplota varu [$^\circ$C]  & 817 & 1635 & 1564 \\\hline
	 Objeven & cca 300 př.n.l. & v 9. stol. př.n.l. & v 1. tis. př.n.l. \\\hline
	 Vzhled & kovově šedý\footnote[frame]{Zdroj: \href{https://commons.wikimedia.org/wiki/File:Arsenic_(33_As).jpg}{Stas1995/Commons}} & stříbrnošedý\footnote[frame]{Zdroj: \href{https://commons.wikimedia.org/wiki/File:Antimony-4.jpg}{Commons}} & stříbrnohnědý\footnote[frame]{Zdroj: \href{https://commons.wikimedia.org/wiki/File:Bismuth_crystals_and_1cm3_cube.jpg}{Alchemist-hp/Commons}} \\
	 &  \begin{minipage}{.2\textwidth}
	 	\adjincludegraphics[width=\linewidth]{img/As.jpg}
	 \end{minipage}
	 	& \begin{minipage}{.2\textwidth}
	 		\adjincludegraphics[width=\linewidth]{img/Sb.jpg}
	 	\end{minipage} & \begin{minipage}{.2\textwidth}
	 	\adjincludegraphics[width=\linewidth]{img/Bi.jpg}
 	\end{minipage} \\\hline
	\end{tabular}
	\vfill
}

\frame{
	\frametitle{}
	\vfill
	\textbf{Moscovium}
	\begin{itemize}
		\item Umělý prvek, protonové číslo 115, Mc.\footnote[frame]{\href{https://iupac.org/iupac-announces-the-names-of-the-elements-113-115-117-and-118/}{IUPAC announces the names of the elements 113, 115, 117, and 118}}
		\item Dřívější název tohoto prvku byl \textit{ununpentium}.
		\item Poprvé byl připraven v roce 2004:\footnote[frame]{\href{https://journals.aps.org/prc/abstract/10.1103/PhysRevC.69.021601}{Experiments on the synthesis of element 115}}
		\item \ce{^{243}_{95}Am + ^{48}_{20}Ca -> ^{288}_{115}Mc + 3 ^1_0n}
		\item \ce{^{243}_{95}Am + ^{48}_{20}Ca -> ^{287}_{115}Mc + 4 ^1_0n}
		\item Předpokládaná elektronová konfigurace: [Rn] 5f$^{14}$ 6d$^{10}$ 7s$^2$ 7p$^3$.
	\end{itemize}

	\begin{center}
	\begin{tabular}{|l|l|l|}
		\hline
		Izotop & Poločas rozpadu & Typ přeměny \\\hline
		$^{286}$Mc & 20 ms & $\alpha$ \\\hline
		$^{287}$Mc & 38 ms & $\alpha$ \\\hline
		$^{288}$Mc & 193 ms & $\alpha$ \\\hline
		$^{289}$Mc & 250 ms & $\alpha$ \\\hline
		$^{290}$Mc & 650 ms & $\alpha$ \\\hline
	\end{tabular}
	\end{center}
	\vfill
}

\section{Chemické a fyzikální vlastnosti}
\frame{
	\frametitle{}
	\vfill
	\begin{itemize}
		\item Prvky 15. skupiny patří mezi nejstarší lidstvu známé prvky.
		\item Arsen a bismut jsou monoizotopické, díky čemuž jsou jejich atomové hmotnosti známy s vysokou přesností.
		\item Antimon má dva stabilní izotopy: $^{121}$Sb (57,25 \%) a $^{123}$Sb (42,75~\%).
		\item Všechny tři prvky vytvářejí několik \textit{allotropních modifikací}.
		\item Elektronová konfigurace valenční slupky všech prvků je ns$^2$ np$^3$.
		\item Běžné oxidační stavy pro arsen a antimon jsou +III a +V.
		\item Oxidační stav V není u bismutu příliš běžný, je to dáno vlivem inertního elektronového páru 6s$^2$. Vytváří sloučeniny převážně v oxidačním stavu III.
		\item U moscovia se očekává, že jako inertní bude vystupovat i orbital 7p takže nejstabilnějším oxidačním stavem bude I.\footnote[frame]{\href{https://doi.org/10.1021/j100612a015}{Predicted Properties of the Superheavy Elements. III. Element 115, Eka-Bismuth}}
		\item Nejběžnější \textit{koordinační čísla} ve sloučeninách těchto prvků jsou 3, 4, 5 a 6.
	\end{itemize}
	\vfill
}

\frame{
	\frametitle{}
	\vfill
	\begin{itemize}
		\item Bismut nemá žádný stabilní izotop.
		\item Dlouho byl za stabilní (a zároveň nejtěžší stabilní jádro) považován izotop $^{209}$Bi, ale v roce 2003 bylo prokázáno, že se rozpadá za uvolnění částice $\alpha$.\footnote[frame]{\href{https://doi.org/10.1038/nature01541}{Experimental detection of alpha-particles from the radioactive decay of natural bismuth}}
		\item \ce{^{209}_{83}Bi -> ^{205}_{81}Tl + ^{4}_{2}He}
		\item Poločas rozpadu je 2,01.10$^{19}$ let.
		\item Za nejtěžší stabilní jádro je nyní považováno jádro $^{208}_{\ 82}$Pb.
	\end{itemize}
	\begin{center}
		\begin{tabular}{|l|l|l|}
			\hline
			\textbf{Izotop} & \textbf{Poločas rozpadu} & \textbf{Typ rozpadu} \\\hline
			$^{207}$Bi & 31,55 let & $\beta^+$ \\\hline
			$^{208}$Bi & 3,7.10$^5$ let & $\beta^+$ \\\hline
			$^{209}$Bi & 2,01.10$^{19}$ let & $\alpha$ \\\hline
			$^{210}$Bi & 5,012 dne & $\beta^-$/$\alpha$ \\\hline
			$^{210m}$Bi & 3,04.10$^6$ let & $\alpha$ \\\hline
		\end{tabular}
	\end{center}
	\vfill
}

\subsection{Allotropní modifikace}
\frame{
	\frametitle{}
	\vfill
	\begin{columns}
		\begin{column}{.75\textwidth}
			\begin{itemize}
				\item Arsen tvoří tři běžné krystalické modifikace. V plynné fázi je ve formě čtyřatomových molekul \ce{As4}.
				\begin{itemize}
					\item Nejběžnější je kovový, \textit{šedý arsen} ($\alpha$ modifikace). Tato modifikace je nestálejší. Je tvořena vrstvami kovalentně vázaného arsenu a má polovodivé vlastnosti.
					\begin{itemize}
						\item V rámci vrstvy sousedí každý arsen se třemi dalšími ve vzdálenosti 251,7~pm a s třemi atomy v sousední vrstvě ve vzdálenosti 312~pm.
					\end{itemize}
					\item \textit{Žlutý arsen} vzniká prudkým ochlazením par arsenu, na světle přechází na šedou modifikaci. Je měkký a voskovitý.\footnote[frame]{\href{https://doi.org/10.1021/acs.chemrev.8b00713}{The Chemistry of Yellow Arsenic}}
					\item \textit{Černý arsen} vzniká ochlazením par arsenu na teplotu 100~$^\circ$C a následnou krystalizací amorfního arsenu v přítomnosti par rtuti.\footnote[frame]{\href{https://doi.org/10.1039/C9NR09627B}{Black arsenic: a new synthetic method by catalytic crystallization of arsenic glass}}
				\end{itemize}
			\end{itemize}
		\end{column}
		\begin{column}{.3\textwidth}
			\begin{figure}
				\adjincludegraphics[width=\textwidth]{img/Yellow-arsenic.jpg}
				\caption*{Struktura žlutého arsenu.}
			\end{figure}
		\end{column}
	\end{columns}
	\vfill
}

\frame{
	\frametitle{}
	\vfill
	\begin{itemize}
		\item Antimon tvoří šest allotropních modifikací.
		\begin{itemize}
			\item \textit{Kovový antimon} ($\alpha$-Sb) je stabilní modifikace, izostrukturní s $\alpha$-As. Má modro-bílou barvu.
			\item \textit{Černý antimon} lze připravit ochlazením par antimonu, je amorfní.
			\item \textit{Žlutý antimon} je stabilní pouze při teplotách pod $-$50~$^\circ$C. Za vyšší teploty se na světle transformuje na stabilnější černý antimon.\footnote[frame]{\href{https://chemistrytalk.org/antimony-element/}{The Lonely Element Antimony}}
			\item Elektrolyticky lze připravit tzv. \textit{explosivní antimon}, který je velmi citlivý na náraz a zahřátí. Připravuje se elektrolýzou roztoku chloridu antimonitého (\ce{SbCl3}) v kyselině chlorovodíkové. Poprvé byl popsán už roku 1855.\footnote[frame]{\href{https://doi.org/10.1002/pssa.2210310118}{On the explosive semiconductor‐semimetal transition of antimony}}
			\item Další dvě modifikace existují pouze za vysokého tlaku. První se připravuje působením tlaku 5~GPa na $\alpha$-Sb, dochází ke změně polohy atomů, každý antimon je obklopen šesti stejně vzdálenými atomy. Další zvýšení tlaku (9~GPa) vede ke vzniku modifikace II, která má nejtěsnější hexagonální uspořádání.
		\end{itemize}
	\end{itemize}
	\vfill
}

\frame{
	\frametitle{}
	\vfill
	\begin{columns}
		\begin{column}{.4\textwidth}
			\begin{figure}
				\adjincludegraphics[width=\textwidth]{img/Antimony.jpg}
				\caption*{Kovový antimon z Mexika.\footnote[frame]{Zdroj: \href{https://commons.wikimedia.org/wiki/File:Antimony_(Mexico)_1_(17152321839).jpg}{James St. John/Commons}}}
			\end{figure}
		\end{column}

		\begin{column}{.7\textwidth}
			\begin{figure}
				\adjincludegraphics[width=\textwidth]{img/Antimon.png}
				\caption*{Antimonový prášek.\footnote[frame]{Zdroj: \href{https://commons.wikimedia.org/wiki/File:Antimon.PNG}{Ondřej Mangl/Commons}}}
			\end{figure}
		\end{column}
	\end{columns}
	\vfill
}

\frame{
	\frametitle{}
	\vfill
	\begin{itemize}
		\item Bismut vytváří několik krystalických modifikací.
		\item Za standardních podmínek je isostrukturní s $\alpha$-As a $\alpha$-Sb (Bi-I).
		\item Při zvýšení tlaku na 2,5~GPa vzniká monoklinická modifikace (Bi-II).
		\item Další zvýšení tlaku na 2,7~GPa vede ke vzniku tetragonálního Bi-III.
		\item Pokud modifikaci Bi-III zahřejeme nad teplotu 170~$^\circ$C získáme orthorombickou modifikaci Bi-IV.\footnote[frame]{\href{https://doi.org/10.1080/08957959.2012.722214}{High-pressure, high-temperature single-crystal study of Bi-IV}}
		\item Zvýšením tlaku na 7,7 GPa získáme kubickou, tělesně centrovanou mřížku Bi-V.\footnote[frame]{\href{https://doi.org/10.1080/08957959.2018.1541456}{High-pressure phase transition of bismuth}}
	\end{itemize}
	\begin{figure}
		\adjincludegraphics[height=0.27\textheight]{img/Bismuth-crystal.jpg}
		\caption*{Krystal bismutu.\footnote[frame]{Zdroj: \href{https://commons.wikimedia.org/wiki/File:Wismut_Kristall_und_1cm3_Wuerfel.jpg}{Alchemist-hp/Commons}}}
	\end{figure}
	\vfill
}

\section{Výskyt a získávání prvků}
\subsection{Arsen}
\frame{
	\frametitle{}
	\vfill
	\begin{itemize}
		\item Koncentrace arsenu v zemské kůře je  1,5--2~ppm.\footnote[frame]{\href{https://www.ncbi.nlm.nih.gov/books/NBK231016/}{Distribution of Arsenic in the Environment}}
		\item Je součástí až 525 minerálů. Nejběžnějším minerálem je arsenopyrit.\footnote[frame]{\href{https://www.mindat.org/element/Arsenic}{The mineralogy of Arsenic}}
		\item Jako zdroj arsenu je nejdůležitější arsenopyrit a další sulfidické minerály.
	\end{itemize}
	\begin{columns}
		\begin{column}{.5\textwidth}
		\begin{center}
	\begin{tabular}{|c|c|}
		\hline
		Minerál & Složení \\
		\hline
		Realgar &  \ce{As4S4} \\
		\hline
		Auripigment & \ce{As2S3} \\
		\hline
		Arsenolit & \ce{As2O3} \\
		\hline
		Loellingit & \ce{FeAs2} \\
		\hline
		Arsenopyrit & \ce{FeAsS} \\
		\hline
		Kobaltin & \ce{CoAsS} \\
		\hline
		Glaukodot & \ce{(Co,Fe)AsS} \\
		\hline
	\end{tabular}
	\end{center}
	\end{column}
	\begin{column}{.5\textwidth}
		\begin{figure}
			\adjincludegraphics[height=0.4\textheight]{img/Arsen.jpg}
			\caption*{Arsen.\footnote[frame]{Zdroj: \href{https://commons.wikimedia.org/wiki/File:Arsen_1.jpg}{Tomihahndorf/Commons}}}
		\end{figure}
	\end{column}
	\end{columns}
	\vfill
}

\subsubsection{Arsenopyrit}
\frame{
	\frametitle{}
	\vfill
	\textbf{Arsenopyrit}
	\begin{itemize}
		\item Arsenopyrit, \ce{FeAsS}, je monoklinický minerál.\footnote[frame]{\href{http://webmineral.com/data/Arsenopyrite.shtml}{Arsenopyrite Mineral Data}}
		\item Často se vyskytuje v přítomnosti zlata, proto se využívá jako indikátor zlatých žil.\footnote[frame]{\href{https://mineraly.sci.muni.cz/sulfidy/arzenopyrit.html}{Atlas minerálů}}
		\item Obsahuje 46 \% arsenu, díky tomu je jeho nejdůležitější rudou.
	\end{itemize}
	\begin{columns}
		\begin{column}{.5\textwidth}
			\begin{figure}
				\adjincludegraphics[width=0.70\textwidth]{img/Arsenopyrite.jpg}
				\caption*{Arsenopyrit, Kutná Hora.\footnote[frame]{Zdroj: \href{https://commons.wikimedia.org/wiki/File:Arsenopyrite_(FeAsS)_(10962863865).jpg}{Jan Helebrant/Commons}}}
			\end{figure}
		\end{column}

		\begin{column}{.5\textwidth}
			\begin{figure}
				\adjincludegraphics[width=0.70\textwidth]{img/Arsenopyrite-117874.jpg}
				\caption*{Arsenopyrit, Mexiko.\footnote[frame]{Zdroj: \href{https://commons.wikimedia.org/wiki/File:Arsenopyrite-117874.jpg}{Robert M. Lavinsky/Commons}}}
			\end{figure}
		\end{column}
	\end{columns}
	\begin{center}
	\end{center}
	\vfill
}

\subsubsection{Auripigment}
\frame{
	\frametitle{}
	\vfill
	\textbf{Auripigment}
	\begin{itemize}
		\item Auripigment, \ce{As2S3}, je jednoklonný minerál.\footnote[frame]{\href{http://webmineral.com/data/Orpiment.shtml}{Orpiment Mineral Data}}$^,$\footnote[frame]{\href{http://mineraly.sci.muni.cz/sulfidy/auripigment.html}{Atlas minerálů}}
		\item Vzniká přeměnou realgaru (\ce{As4S4}) nebo hydrotermálně.
		\item Byl využíván už v Římské říši, po staletí byl používán jako pigment.
	\end{itemize}
	\begin{columns}
		\begin{column}{.5\textwidth}
			\begin{figure}
				\adjincludegraphics[height=0.32\textheight]{img/Baryte-Orpiment-61034.jpg}
				\caption*{Auripigment a baryt, Rusko.\footnote[frame]{Zdroj: \href{https://commons.wikimedia.org/wiki/File:Baryte-Orpiment-61034.jpg}{Robert M. Lavinsky/Commons}}}
			\end{figure}
		\end{column}

		\begin{column}{.5\textwidth}
			\begin{figure}
				\adjincludegraphics[height=0.32\textheight]{img/Orpiment-117492.jpg}
				\caption*{Auripigment, Rusko.\footnote[frame]{Zdroj: \href{https://commons.wikimedia.org/wiki/File:Arsenopyrite-117874.jpg}{Robert M. Lavinsky/Commons}}}
			\end{figure}
		\end{column}
	\end{columns}
	\vfill
}

\subsubsection{Realgar}
\frame{
	\frametitle{}
	\vfill
	\begin{columns}
		\begin{column}{.7\textwidth}
			\textbf{Realgar}
			\begin{itemize}
				\item Realgar, \ce{As4S4}, je jednoklonný minerál.\footnote[frame]{\href{http://webmineral.com/data/Realgar.shtml}{Realgar Mineral Data}}
				\item Vzniká rozkladem jiných sulfidů, hlavně arsenopyritu.
				\item Na světle se pomalu rozkládá na směs auripigmentu (\ce{As2S3}) a oxidu arsenitého.\footnote[frame]{\href{https://mineraly.sci.muni.cz/sulfidy/realgar.html}{Atlas minerálů}}
				\item Je částečně rozpustný v kyselinách a hydroxidu draselném.
			\end{itemize}
		\end{column}

		\begin{column}{.35\textwidth}
			\begin{figure}
				\adjincludegraphics[width=\textwidth]{img/Realgar-Calcite-37467.jpg}
				\caption*{Realgar na kalcitu.\footnote[frame]{Zdroj: \href{https://commons.wikimedia.org/wiki/File:Realgar-Calcite-37467.jpg}{Robert M. Lavinsky/Commons}}}
			\end{figure}
		\end{column}
	\end{columns}
	\vfill
}

\subsubsection{Arsenolit}
\frame{
	\frametitle{}
	\vfill
	\textbf{Arsenolit}
	\begin{itemize}
		\item Arsenolit, \ce{As2O3, As4O6}, je kubický, vysoce toxický minerál.\footnote[frame]{\href{https://www.mindat.org/min-294.html}{Arsenolite}}
		\item Vzniká oxidací (zvětráváním) sulfidických minerálů arsenu.
		\item Je rozpustný ve vodě.\footnote[frame]{\href{http://mineraly.sci.muni.cz/oxidy/arsenolit.html}{Atlas minerálů}}
	\end{itemize}
	\begin{columns}
		\begin{column}{.5\textwidth}
			\begin{figure}
				\adjincludegraphics[width=0.75\textwidth]{img/Arsenolite-333170.jpg}
				\caption*{Arsenolit, USA.\footnote[frame]{Zdroj: \href{https://commons.wikimedia.org/wiki/File:Arsenolite-333170.jpg}{Robert M. Lavinsky/Commons}}}
			\end{figure}
		\end{column}

		\begin{column}{.5\textwidth}
			\begin{figure}
				\adjincludegraphics[width=0.75\textwidth]{img/Arsenolite-90693.jpg}
				\caption*{Arsenolit, Německo.\footnote[frame]{Zdroj: \href{https://commons.wikimedia.org/wiki/File:Arsenolite-90693.jpg}{Robert M. Lavinsky/Commons}}}
			\end{figure}
		\end{column}
	\end{columns}
	\vfill
}

\subsubsection{Dobšináit}
\frame{
	\frametitle{}
	\vfill
	\textbf{Dobšináit}
	\begin{itemize}
		\item Dobšináit, \ce{Ca2[AsO4]2.H2O}, je minerál objevený roku 2021 na Slovensku.\footnote[frame]{\href{https://www.czechsight.cz/dobsinait-nove-definovany-mineral-z-uzemi-slovenska/}{Dobšináit, nově definovaný minerál z území Slovenska}}
		\item Jedná se 23. původní minerál popsaný na slovenském území.
		\item Název je odvozen od města Dobšiná, v jehož blízkosti byl minerál objeven.\footnote[frame]{\href{https://www.sav.sk/?lang=sk&doc=services-news&source_no=20&news_no=9364}{Slovensko dalo svetu nový minerál – dobšináit}}
		\item Na objevu se podíleli i geologové z MU.\footnote[frame]{\href{https://www.sci.muni.cz/clanky/brnensti-vedci-se-podileli-na-objevu-noveho-mineralu}{Brněnští vědci se podíleli na objevu nového minerálu}}
		\item Má bílou barvu, díky příměsím kobaltu bývá zbarven do světle růžové barvy.\footnote[frame]{\href{https://ct24.ceskatelevize.cz/veda/3270590-brnensti-vedci-popsali-novy-mineral-dobsinait-nasli-na-vychodnim-slovensku}{Brněnští vědci popsali nový minerál. Dobšináit našli na východním Slovensku}}
	\end{itemize}
	\vfill
}

\subsubsection{Výroba arsenu}
\frame{
	\frametitle{}
	\vfill
	\begin{itemize}
		\item Kovový arsen lze připravit žíháním \ce{FeAs2} nebo FeAsS bez přístupu vzduchu, arsen sublimuje a lze ho tedy snadno izolovat sublimací ve vakuu nebo ve vodíkové atmosféře.
		\item \ce{FeAsS ->[700\ $^\circ$C] FeS + As(g)}
		\item Pražením na vzduchu vzniká přímo arsenik (\ce{As2O3}), který také sublimuje.
		\item \ce{FeAsS + 5 O2 -> Fe2O3 + As2O3 + SO2}
		\item Ten lze získat i jako vedlejší produkt při tavení koncentrátů mědi, zlata a~olova.\footnote[frame]{\href{https://www.usgs.gov/centers/nmic/arsenic-statistics-and-information}{Arsenic Statistics and Information}}
		\item Hlavním světovým producentem arseniku je Čína.\footnote[frame]{\href{https://pubs.usgs.gov/periodicals/mcs2020/mcs2020-arsenic.pdf}{Arsenic Data Sheet - Mineral Commodity Summaries 2020}}
	\end{itemize}
	\vfill
}

\subsection{Antimon}
\frame{
	\frametitle{}
	\vfill
	\begin{itemize}
		\item V zemské kůře je jeho koncentrace jen 0,2--0,5 ppm, v mořské vodě 0,3~$\mu$g.l$^{-1}$.\footnote[frame]{\href{https://www.usgs.gov/centers/nmic/antimony-statistics-and-information}{Antimony Statistics and Information}}
		\item Je součástí více než 280 minerálů.\footnote[frame]{\href{https://www.mindat.org/element/Antimony}{The mineralogy of Antimony}}
		\item Nejdůležitější rudou je antimonit.
	\end{itemize}

	\begin{center}
	\begin{tabular}{|c|c|}
		\hline
		Minerál & Složení \\
		\hline
		Stibnit/Antimonit & \ce{Sb2S3} \\
		\hline
		Livingstonit & \ce{HgSb4S8} \\
		\hline
		Ullmanit & \ce{NiSbS} \\
		\hline
		Jamesonit & \ce{FePb4Sb6S14} \\
		\hline
		Tetraedit & \ce{Cu3SbS3} \\
		\hline
		Wolfsbergit & \ce{CuSbS2} \\
		\hline
		Valentinit & \ce{Sb2O3} \\
		\hline
	\end{tabular}
	\end{center}
	\vfill
}

\subsubsection{Antimonit}
\frame{
	\frametitle{}
	\vfill
	\textbf{Antimonit (stibnit)}
	\begin{itemize}
	\item Antimonit, \ce{Sb2S3}, je kosočtverečný minerál.\footnote[frame]{\href{https://www.mindat.org/min-3782.html}{Stibnite}} Dříve byl označován jako \textit{leštěnec antimonový}.
	\item Vzniká na nízkoteplotních hydrotermálních žilách.\footnote[frame]{\href{http://mineraly.sci.muni.cz/sulfidy/antimonit.html}{Atlas minerálů}}
	\item Je rozpustný v \ce{HNO3} a horké HCl, v KOH černá.
	\end{itemize}
	\begin{columns}
		\begin{column}{.5\textwidth}
			\begin{figure}
				\adjincludegraphics[height=0.3\textheight]{img/Stibnite.jpg}
				\caption*{Antimonit.\footnote[frame]{Zdroj: \href{https://commons.wikimedia.org/wiki/File:Stibnite.jpg}{Pepperedjane/Commons}}}
			\end{figure}
		\end{column}
		\begin{column}{.5\textwidth}
			\begin{figure}
				\adjincludegraphics[height=0.35\textheight]{img/Stibnite-21985.jpg}
				\caption*{Antimonit, Rumunsko.\footnote[frame]{Zdroj: \href{https://commons.wikimedia.org/wiki/File:Stibnite-196786.jpg}{Robert M. Lavinsky/Commons}}}
			\end{figure}
		\end{column}
	\end{columns}
	\vfill
}

\subsubsection{Výroba antimonu}
\frame{
	\frametitle{}
	\vfill
	\begin{itemize}
		\item Čína vyrábí více než 60~\% světové produkce antimonu.\footnote[frame]{\href{https://www.usgs.gov/centers/nmic/antimony-statistics-and-information}{Antimony Statistics and Information}}
		\item V roce 2020 bylo vytěženo 111 000 tun rud antimonu.
		\item Sulfidické rudy jsou pražením na vzduchu převedeny na oxid:\footnote[frame]{\href{https://www.researchgate.net/publication/299133344_Antimony_Production_and_Commodites}{Antimony Production and Commodites}}
		\item \ce{2 Sb2S3 + 9 O2 ->[1000 $^\circ$C] 2 Sb2O3 + 6 SO2}
		\item Oxid se poté redukuje uhlím:
		\item \ce{Sb2O3 + 3 CO -> 2 Sb + 3 CO2}
		\item \ce{3 CO2 + 3 C -> 6 CO}
		\item Sulfid je možno redukovat i přímo pomocí železa:
		\item \ce{Sb2S3 + 3 Fe -> 2 Sb + 3 FeS}
		\item Vysoce čistý antimon pro polovodičové aplikace se připravuje chemickou redukcí sloučenin s vysokou čistotou, např. \ce{SbCl3}.
		\item \ce{2 SbCl3 + 3 H2O ->[][-6 HCl] Sb2O3 + 3 H2 -> 2 Sb + 3 H2O}
	\end{itemize}
	\vfill
}

\frame{
	\frametitle{}
	\vfill
	\begin{figure}
		\begin{figure}
			\adjincludegraphics[height=0.7\textheight]{img/Antimony_-_world_production_trend.png}
			\caption*{Světová produkce antimonu v letech 1900--2010.\footnote[frame]{Zdroj: \href{https://commons.wikimedia.org/wiki/File:Antimony_-_world_production_trend.svg}{Leyo/Commons}}}
		\end{figure}
	\end{figure}
	\vfill
}

\subsection{Bismut}
\frame{
	\frametitle{}
	\vfill
	\begin{itemize}
		\item V zemské kůře je jeho koncentrace jen 0,2 ppm, v mořské vodě 0,02~$\mu$g.l$^{-1}$.\footnote[frame]{\href{https://www.usgs.gov/centers/nmic/bismuth-statistics-and-information}{Bismuth Statistics and Information}}
		\item Je součástí zhruba 200 minerálů.
		\item Nejdůležitějšími rudami jsou bismit (\ce{Bi2O3}) a bismutin (\ce{Bi2S3}).
		\item Kovový bismut nacházíme v Německu, Austrálii, Japonsku i v Česku (Jáchymov).\footnote[frame]{\href{https://rruff.info/doclib/hom/bismuth.pdf}{Bismuth}}
	\end{itemize}
	\begin{center}
		\begin{tabular}{|c|c|}
			\hline
			Minerál & Složení \\
			\hline
			Bismut & \ce{Bi} \\
			\hline
			Bismit & \ce{Bi2O3} \\
			\hline
			Bismutin & \ce{Bi2S3} \\
			\hline
			Bismutit & \ce{Bi2(CO3)O2} \\
			\hline
			Galenobismutit & \ce{PbBi2S4} \\
			\hline
			Bismutocolumbit & \ce{Bi(Nb,Ta)O4} \\
			\hline
			Francisit & \ce{Cu3Bi(SeO3)2O2Cl} \\
			\hline
		\end{tabular}
	\end{center}
	\vfill
}

\frame{
	\frametitle{}
	\vfill
	\begin{itemize}
		\item Hlavním zdrojem bismutu jsou vedlejší produkty výroby olova, zinku a mědi.
		\item Ryzí bismut se vyskytuje v Austrálii, Bolívii a Číně.
		\item Sulfidické rudy se nejprve praží na vzduchu a poté se redukují železem nebo uhlíkem (dřevěným uhlím).
		\item Důležitým zdrojem je i \textit{recyklace}.
		\item V 70. letech 20. století cena bismutu prudce vzrostla, šlo o reakci na zvýšenou poptávku po bismutu jako aditiva do hliníku a~ocelí.
	\end{itemize}

	\begin{figure}
		\adjincludegraphics[height=0.3\textheight]{img/BrokenBismuthIngot.jpg}
		\caption*{Ingot bismutu.\footnote[frame]{Zdroj: \href{https://commons.wikimedia.org/wiki/File:BrokenBismuthIngot.jpg}{Unconventional2/Commons}}}
	\end{figure}
	\vfill
}

\frame{
	\frametitle{}
	\vfill
	\begin{figure}
		\adjincludegraphics[height=0.65\textheight]{img/BiPrice.png}
		\caption*{Světová produkce a cena bismutu.\footnote[frame]{Zdroj: \href{https://commons.wikimedia.org/wiki/File:BiPrice.png}{Materialscientist/Commons}}}
	\end{figure}
	\vfill
}

\frame{
	\frametitle{}
	\vfill
	\begin{itemize}
		\item Rudy bismutu se těží v Bolívii\footnote[frame]{\href{https://www.mindat.org/loc-40745.html}{Tazna Mine}} a~Číně.\footnote[frame]{\href{https://www.mindat.org/loc-48873.html}{Fankou Mine}}
		\item Hlavním producentem bismutu je Čína, kde se získává jako vedlejší produkt při výrobě wolframu.
	\end{itemize}
	\begin{figure}
		\adjincludegraphics[width=0.5\textwidth]{img/BiRedukce.jpg}
		\caption*{Redukce oxidu bismutitého vodíkem.}
	\end{figure}
	\vfill
}

\subsubsection{Bismit}
\frame{
	\frametitle{}
	\vfill
	\begin{columns}
		\begin{column}{.6\textwidth}
			\textbf{Bismit}
			\begin{itemize}
				\item Bismit, \ce{Bi2O3}, je jednoklonný, žlutozelený  minerál.\footnote[frame]{\href{https://www.mindat.org/min-682.html}{Bismite}}
				\item Vzniká jako oxidační produkt minerálů bismutu.\footnote[frame]{\href{http://webmineral.com/data/Bismite.shtml}{Bismite Mineral Data}}
				\item Poprvé byl popsán v roce 1868 v Nevadě.
			\end{itemize}
		\end{column}
		\begin{column}{.45\textwidth}
			\begin{figure}
				\adjincludegraphics[width=\textwidth]{img/Bismite-Pucherite-sf39a.jpg}
				\caption*{Bismit, Německo.\footnote[frame]{Zdroj: \href{https://commons.wikimedia.org/wiki/File:Bismite-Pucherite-sf39a.jpg}{Robert M. Lavinsky/Commons}}}
			\end{figure}
		\end{column}
	\end{columns}
	\vfill
}

\subsubsection{Bismutin}
\frame{
	\frametitle{}
	\vfill
	\begin{columns}
		\begin{column}{.7\textwidth}
			\textbf{Bismutin}
			\begin{itemize}
				\item Bismutinit (bismutin, leštěnec bismutový), \ce{Bi2S3}, je kosočtverečný minerál.\footnote[frame]{\href{https://www.mindat.org/min-686.html}{Bismuthinite}}$^,$\footnote[frame]{\href{http://geologie.vsb.cz/loziska/suroviny/rudy/bismutin.html}{Bismutin}}
				\item Vzniká v hydrotermálních žilách.\footnote[frame]{\href{http://www.handbookofmineralogy.org/pdfs/bismuthinite.pdf}{Bismuthinite}}
				\item Poprvé byl popsán v roce 1832 v bolívijských dolech.
			\end{itemize}
		\end{column}
		\begin{column}{.35\textwidth}
			\begin{figure}
				\adjincludegraphics[width=0.9\textwidth]{img/Bismuthinite-136224.jpg}
				\caption*{Bismutin, Bolívie.\footnote[frame]{Zdroj: \href{https://commons.wikimedia.org/wiki/File:Bismuthinite-136224.jpg}{Robert M. Lavinsky/Commons}}}
			\end{figure}
		\end{column}
	\end{columns}
	\vfill
}

\subsubsection{Heyrovskýit}
\frame{
	\frametitle{}
	\vfill
	\begin{columns}
		\begin{column}{.6\textwidth}
			\textbf{Heyrovskýit}
			\begin{itemize}
				\item Heyrovskýit, \ce{Pb6Bi2S9}, je cínově bílý, až šedý kosočtverečný minerál.\footnote[frame]{\href{https://www.mindat.org/min-6993.html}{Heyrovskýite}}
				\item Vzniká v hydrotermálních žilách.
				\item Poprvé byl popsán v západních Čechách.
				\item Jedná se o vzácný minerál, proto nemá komerční využití.
			\end{itemize}
		\end{column}
		\begin{column}{.45\textwidth}
			\begin{figure}
				\adjincludegraphics[width=\textwidth]{img/Heyrovskýite_-_quartz.jpg}
				\caption*{Heyrovskýit na křemeni, Rakovník.\footnote[frame]{Zdroj: \href{https://commons.wikimedia.org/wiki/File:Heyrovskýite_-_quartz.jpg}{Marc Gravel/Commons}}}
			\end{figure}
		\end{column}
	\end{columns}
	\vfill
}

\section{Využití prvků}
\subsection{Arsen}
\frame{
	\frametitle{}
	\vfill
	\begin{itemize}
		\item Hlavní využití arsenu je v slitinách s olovem a mědí.
		\item Malý podíl arsenu v slitině Pb/Sb zlepšuje vlastnosti mřížek používaných v akumulátorech.
		\item Arsen (0,5--2,0~\%) také zlepšuje kulatost olověných broků.
		\item Sloučeniny arsenu se využívají v zemědělství jako herbicidy.
		\item Oxid arsenitý se používá k odbarvování lahvového skla.
		\item Z arsenitanu sodného se připravovaly lázně na odhmyzování ovcí a hovězího dobytka.
	\end{itemize}
	\begin{figure}
		\adjincludegraphics[width=.7\textwidth]{img/As-herbicidy.png}
		\caption*{Methylarseničnan monosodný a kyselina dimethylarseničná se využívají jako herbicidy.}
	\end{figure}
	\vfill
}

\frame{
	\frametitle{}
	\vfill
	\begin{itemize}
		\item \textbf{GaAs} je polovodič III/V používaný pro konstrukci PN přechodů v různých typech tranzistorů, díky vlastnostem GaAs mohou tyto tranzistory pracovat až do frekvence 250~GHz.
		\item Využívá se také při konstrukci fotovoltaických článků s vysokou účinností, např. pro vesmírné sondy.
		\item Má strukturu sfaleritu. Vyrábí se několika metodami:
		\item VGF (Vertical Gradient Freeze) procesem -- tavenina je ve válcovém kelímku postupně ochlazována.\footnote[frame]{\href{https://arxiv.org/pdf/2002.11447.pdf}{Control of the Vertical Gradient Freeze crystal growth process via backstepping}}
		\item Pomocí Czochralskiho metody.\footnote[frame]{\href{https://doi.org/10.1016/0022-0248(73)90061-4}{Liquid-seal Czochralski growth of gallium arsenide}}
		\item MOCVD:\footnote[frame]{\href{https://doi.org/10.1016/0042-207X(90)93833-5}{Mechanism of gallium arsenide MOCVD}}
		\begin{itemize}
			\item \ce{Ga(CH3)3 + AsH3 -> GaAs + 3 CH4}
		\end{itemize}
	\end{itemize}
	\vfill
}

\frame{
	\frametitle{}
	\vfill
	\begin{columns}
		\begin{column}{.5\textwidth}
			\begin{figure}
				\adjincludegraphics[width=.95\textwidth]{img/Gallium-arsenide-unit-cell-3D-balls.png}
				\caption*{Krystalová struktura GaAs.\footnote[frame]{Zdroj: \href{https://commons.wikimedia.org/wiki/File:Gallium-arsenide-unit-cell-3D-balls.png}{Benjah-bmm27/Commons}}}
			\end{figure}
		\end{column}
		\begin{column}{.5\textwidth}
			\begin{figure}
				\adjincludegraphics[width=.8\textwidth]{img/MC331694PDTA.jpg}
				\caption*{Zesilovač z GaAs.\footnote[frame]{Zdroj: \href{https://commons.wikimedia.org/wiki/File:MC331694PDTA.jpg}{Epop/Commons}}}
			\end{figure}
		\end{column}
	\end{columns}
	\vfill
}

\subsection{Antimon}
\frame{
	\frametitle{}
	\vfill
	\begin{itemize}
		\item Antimon se využívá nejvíce jako zpomalovač hoření, dále jako součást slitin pro baterie, ložiska a pájky.
		\begin{itemize}
			\item Přídavek antimonu do elektrod olověných akumulátoru zlepšuje jejich pevnost a nabíjecí charakteristiky.
			\item Některé bezolovnaté pájky obsahují antimon.
		\end{itemize}
		\item Oxid antimonitý má ve spojení s halogenovanými sloučeniny výrazné zhášecí účinky.
		\item Velice čistý antimon se využívá v polovodičovém průmyslu.
		\begin{itemize}
			\item ZnSb vykazuje termoelektrické vlastnosti.
			\item GaSb a InSb se využívají pro konstrukci detektorů a emitorů infračerveného záření, např. pro IR LED, tranzistory a LASERy.
		\end{itemize}
	\end{itemize}
	\vfill
}

\frame{
	\frametitle{}
	\vfill
	\textbf{Liteřina}
	\begin{columns}
		\begin{column}{.65\textwidth}
			\begin{itemize}
				\item Liteřina, neboli písmový kov, je snadno tavitelná, ale stále dostatečně tvrdá slitina využívaná k odlévání tiskových písmen pro ruční sazbu.\footnote[frame]{\href{https://www.encyklopedieknihy.cz/index.php/Písmový_kov}{Písmový kov}}
				\item Její přibližné složení je: 50-86~\% Pb,\\ 3-20~\% Sn a 11-30~\% Sb.
				\item Využívala se od 15. století, její význam poklesl ve 20. století, s nástupem modernějších technik tisku.
			\end{itemize}
		\end{column}
		\begin{column}{.15\textwidth}
			\begin{figure}
				\adjincludegraphics[height=.35\textheight]{img/Garamond_type_fi-ligature_2.jpg}
				\caption*{Ligatura fi odlitá z liteřiny.\footnote[frame]{Zdroj: \href{https://commons.wikimedia.org/wiki/File:Garamond_type_ſi-ligature_2.jpg}{Daniel Ullrich/Commons}}}
			\end{figure}
		\end{column}
		\begin{column}{.25\textwidth}
			\begin{figure}
				\adjincludegraphics[width=\textwidth]{img/Fotothek0008535.jpg}
				\caption*{Výroba liter.\footnote[frame]{Zdroj: \href{https://commons.wikimedia.org/wiki/File:Fotothek\%20df\%20tg\%200008535\%20St\%C3\%A4ndebuch\%20\%5E\%20Beruf\%20\%5E\%20Handwerk\%20\%5E\%20Gie\%C3\%9Fer\%20\%5E\%20Letter.jpg}{Deutsche Fotothek/Commons}}}
			\end{figure}
		\end{column}
	\end{columns}
	\vfill
}

\subsection{Bismut}
\frame{
	\frametitle{}
	\vfill
		\begin{itemize}
			\item Bismut má pouze několik komerčních aplikací:
			\item Díky podobné hustotě a nízké toxicitě ho lze využít jako náhradu olova ve střelivu, rybářských olůvcích a podobných aplikacích.
			\item Oxid-chlorid bismutitý (\ce{BiOCl}) se využívá jako žlutý pigment v kosmetice.\footnote[frame]{\href{https://www.cosmeticsandtoiletries.com/cosmetic-ingredients/colorant/article/21833876/bismuth-oxychloridea-multifunctional-color-additive}{Bismuth Oxychloride-A Multifunctional Color Additive}}
			\item Oxid bismutitý se využívá ve sklářském a keramickém průmyslu a také při výrobě katalyzátorů a magnetů.\footnote[frame]{\href{https://doi.org/10.1179/1743280412Y.0000000010}{Review of \ce{Bi2O3} based glasses for electronics and related applications}}
			\item Tellurid bismutitý, \ce{Bi2Te3}, má termoelektrické vlastnosti, využívá se při konstrukci Peltierových článků pro mobilní ledničky, chladiče CPU a také jako detektor NIR záření.
		\end{itemize}
	\vfill
}

\frame{
	\frametitle{}
	\textbf{Nízkotající slitiny}
	\begin{tabular}{|l|l|l|}
		\hline
		\textbf{Slitina} & \textbf{Obchodní název} & \textbf{Teplota tání [$^\circ$C]} \\
		\hline
		Bi45Pb23Sn8In19Cd5 & Slitina 47  &  47 \\
		\hline
		Bi49Pb18Sn12In21 & Slitina 58  & 58 \\
		\hline
		Bi50Pb27Sn13Cd & Woodův kov 1 & 68-72 \\
		\hline
		Bi50Pb25Sn12Cd & Woodův kov 2 & 60-64 \\
		\hline
		Bi50Sn25Pb & Roseův kov & 92-96 \\
		\hline
		Bi55Pb32Sn & Molotův kov & 96-98 \\
		\hline
		Bi74Pb7Sn & Biola 1 & 104-463 \\
		\hline
		Bi50Pb43Cd & Biola 3 & 80-84 \\
		\hline
		Bi8Sn57Pb & Stabia 1 & 139-178 \\
		\hline
		Pb45Bi10Sn & Stabia 4 & 97-169 \\
		\hline
		Pb25Bi25Sn & Stabia 6 & 97-161 \\
		\hline
		Pb73Bi23Sn3Zn & Plumbia 3 & 183-224 \\
		\hline
		Pb57Bi8Sn & Plumbia 5 & 174-214 \\
		\hline
	\end{tabular}
}

\section{Sloučeniny}
\subsection{Hydridy}
\frame{
	\textbf{Hydridy prvků 15. skupiny}
	\begin{center}
		\begin{tabular}{|l|l|l|l|l|l|}
			\hline
			\textbf{Vzorec} & \textbf{Název} & \textbf{T$_t$ [$^\circ$C]} & \textbf{T$_v$ [$^\circ$C]} & Vzd. \ce{E-H} [pm] & Úhel \ce{H-E-H} \\\hline
			\ce{NH3} & Amoniak & $-$78 & $-$33 & 101,2 & 106,7 \\\hline
			\ce{PH3} & Fosfan & $-$133 & $-$88 & 142,0 & 93,3 \\\hline
			\ce{AsH3} & Arsan & $-$116 & $-$63 & 151,1 & 92,1 \\\hline
			\ce{SbH3} & Stiban & $-$88 & $-$17 & 170,4 & 91,6 \\\hline
			\ce{BiH3} & Bismutan & - & 17 & - & 90 \\\hline
		\end{tabular}
	\end{center}
	\vfill
}

\frame{
	\frametitle{}
	\vfill
	\begin{itemize}
		\item Všechny tři hydridy jsou nestabilní, jedovaté, bezbarvé plyny.
		\item Na rozdíl od amoniaku a fosfanu nejeví hydridy tendenci ke tvorbě oniových iontů \ce{MH_4^+}.
		\item Arsan a stiban jsou podobné fosfanu, ale ochotněji se rozkládají na prvky.
		\item Oba jsou extrémně toxické plyny.
		\item \textit{Arsan}, \ce{AsH3}, vzniká redukcí sloučenin arsenu nascentním vodíkem, viz \ce{As2O3}. Nebo redukcí chloridu arsenitého hydridy.\footnote[frame]{\href{https://doi.org/10.1021/ic50068a024}{Synthesis of the hydrides of germanium, phosphorus, arsenic, and antimony by the solid-phase reaction of the corresponding oxide with lithium aluminum hydride}}
		\item \ce{AsCl3 + 3 LiAlH4 ->[Et2O] AsH3 + 3 LiCl + 3 AlH3}
		\item Využívá se k výrobě GaAs.
		\item Lze jej přímo oxidovat kyslíkem na oxid arsenitý:
		\item \ce{2 AsH3 + 3 O2 -> As2O3 + 3 H2O}
	\end{itemize}
	\vfill
}

\frame{
	\frametitle{}
	\vfill
	\begin{itemize}
		\item \textit{Stiban}, \ce{SbH3}, zapáchá po zkažených vejcích.
		\item Připravuje se podobně jako arsan:
		\item \ce{4 SbCl3 + 3 NaBH4 -> 4 SbH3 + 3 NaCl + 3 BCl3}
		\item \ce{SbO$_3^{3-}$ + 3 Zn + 9 H3O+ -> SbH3 + 3 Zn^{2+} + 12 H2O}
		\item Za pokojové teploty se autokatalyticky rozkládá (až explozivně):
		\item \ce{2 SbH3 -> 3 H2 + 2 Sb}
		\item Využívá se v polovodičovém průmyslu k dopování křemíku antimonem.
		\item \textit{Bismutan}, \ce{BiH3}, je nestabilní, rozkládá se již pod teplotou 0~$^\circ$C.\footnote[frame]{\href{https://doi.org/10.1021/jp034379y}{Infrared Spectra of Antimony and Bismuth Hydrides in Solid Matrixes}}
		\item Můžeme jej připravit disproporcionací \ce{MeBiH2} nebo \ce{Me2BiH}:
		\item \ce{3 MeBiH2 ->[$-45~^\circ$C] 2 BiH3 + BiMe3}
		\item Redukce pomocí tetrahydridoboritanu není v tomto případě použitelná.
	\end{itemize}
	\vfill
}

\subsection{Oxidy a kyseliny}
%As2O3, As2O5
%Sb2O3, Sb2O4, Sb2O5
%Bi2O3, Bi2O5
\frame{
	\frametitle{}
	\vfill
	\begin{itemize}
		\item \textit{Oxid arsenitý}, \ce{As2O3}, dříve označovaný jako \textit{arsenik} je bílý, toxický prášek. Je to nejdůležitější sloučenina arsenu.
		\item Získává se převážně třemi postupy:
		\begin{itemize}
			\item Spalováním arsenu na vzduchu:
			\item \ce{4 As + 3 O2 -> 2 As2O3}
			\item Hydrolýzou chloridu:
			\item \ce{2 AsCl3 + 3 H2O -> As2O3 + 6 HCl}
			\item Pražením sulfidických rud:
			\item \ce{2 As2S3 + 9 O2 -> 2 As2O3 + 6 SO2}
		\end{itemize}
		\item V plynném stavu je tvořen molekulami \ce{As4O6}.
		\item Stejná struktura je přítomna i v krystalickém oxidu v kubické formě.
		\item Kromě toho známe i monoklinickou formu tvořenou polymerními jednotkami \ce{\{As2O3\}}.
	\end{itemize}
	\vfill
}

\frame{
	\frametitle{}
	\vfill
	\begin{columns}
		\begin{column}{.5\textwidth}
			\begin{figure}
				\adjincludegraphics[height=.5\textheight]{img/Arsenic_trioxide.jpg}
				\caption*{Oxid arsenitý.\footnote[frame]{Zdroj: \href{https://commons.wikimedia.org/wiki/File:Arsenic_trioxide.jpg}{Walkerma/Commons}}}
			\end{figure}
		\end{column}
		\begin{column}{.5\textwidth}
			\begin{figure}
				\adjincludegraphics[height=.5\textheight]{img/As4O6-molecule-from-arsenolite-xtal-3D-balls.png}
				\caption*{Molekula \ce{As2O3}.\footnote[frame]{Zdroj: \href{https://commons.wikimedia.org/wiki/File:Arsenic-trioxide-3D-balls.png}{Benjah-bmm27/Commons}}}
			\end{figure}
		\end{column}
	\end{columns}
	\vfill
}

\frame{
	\frametitle{}
	\vfill
	Oxid arsenitý je výchozí sloučeninou pro mnoho dalších sloučenin arsenu.
	\begin{figure}
		\adjincludegraphics[height=.8\textheight]{img/As2O3-RX.png}
	\end{figure}
	\vfill
}

\frame{
	\frametitle{}
	\vfill
	\begin{columns}
		\begin{column}{.7\textwidth}
			\begin{itemize}
				\item Otrava arsenikem se dokazovala pomocí \textbf{Marshovy zkoušky}.\footnote[frame]{\href{https://www.youtube.com/watch?v=-vUZdAwgl2g}{Detecting Arsenic - The Marsh-Test}}
				\item Autorem testu je britský chemik James Marsh, který tuto metodu publikoval v roce 1836.\footnote[frame]{\href{https://archive.org/details/edinburghnewphil21edin/page/228/mode/2up?view=theater}{Account of a method of separating small quantities of arsenic from substances with which it may be mixed}}
				\item Ke vzorku se přidá kyselina sírová a zinek (beze stop arsenu).
				\item Reakcí zinku s kyselinou vzniká tzv. \textit{nascentní vodík}, který zredukuje oxid arseničný na arsan.
				\item \ce{As2O3 + 6 Zn + 6 H2SO4 -> 2 AsH3 + 6 ZnSO4 + 3 H2O}
				\item Vzniklý arsan je, po vyčištění a vysušení, veden skleněnou trubicí zahřívanou kahanem, kde se rozkládá za vzniku arsenového zrcátka.
			\end{itemize}
		\end{column}
	\begin{column}{.3\textwidth}
		\begin{figure}
			\adjincludegraphics[height=.5\textheight]{img/Marsh_James.jpg}
			\caption*{James Marsh, 1794-1846.\footnote[frame]{Zdroj: \href{https://commons.wikimedia.org/wiki/File:Marsh_James.jpg}{W. T. Vincent/Commons}}}
		\end{figure}
	\end{column}
	\end{columns}
	\vfill
}

\frame{
	\frametitle{}
	\vfill
	\begin{figure}
		\adjincludegraphics[width=.85\textwidth]{img/Marsh_test_apparatus.jpg}
		\caption*{Aparatura na Marshovu zkoušku.\footnote[frame]{Zdroj: \href{https://commons.wikimedia.org/wiki/File:Marsh_test_apparatus.jpg}{Hugh McMuigan/Commons}}}
	\end{figure}
	\vfill
}

\frame{
	\frametitle{}
	\vfill
	\begin{columns}
		\begin{column}{.5\textwidth}
			\begin{itemize}
				\item Tento důkaz byl použit v roce 1840 při soudním řízení s Marií Lafarge, podezřelou z otrávení manžela arsenikem.\footnote[frame]{\href{https://www.britishmuseum.org/collection/term/BIOG228917}{Marie Lafarge}}
				\item Marshův test provedl španělský toxikolog Mathieu Joseph Bonaventure Orfila.\footnote[frame]{\href{https://www.nlm.nih.gov/exhibition/visibleproofs/galleries/biographies/orfila.html}{Mathieu Joseph Bonaventure Orfila}}
				\item Událost byla zachycena v jednom z dílů seriálu \textit{Dobrodružství kriminalistiky}.\footnote[frame]{\href{https://www.ceskatelevize.cz/porady/898901-dobrodruzstvi-kriminalistiky/289310910380002-jed/}{Dobrodružství kriminalistiky}}
			\end{itemize}
		\end{column}
		\begin{column}{.5\textwidth}
			\begin{figure}
				\adjincludegraphics[width=.7\textwidth]{img/Lafarge.jpg}
				\caption*{Marie Lafarge.\footnote[frame]{Zdroj: \href{https://commons.wikimedia.org/wiki/File:Lafarge.jpg}{Commons}}}
			\end{figure}
		\end{column}
	\end{columns}
	\vfill
}

\frame{
	\frametitle{}
	\vfill
		\begin{itemize}
			\item Arsen je možno dokázat i \textit{Gutzeitovým testem}.\footnote[frame]{\href{https://doi.org/10.1039/AN9012600181}{The Gutzeit test for arsenic}}
			\item Jeho autorem je německý chemik Max Adolf Gutzeit (1847--1915).
			\item Arsenovodík připravíme podobně jako u Marshova testu:
			\item \ce{6 Zn + 12 HCl + As2O3 -> 2 AsH3 + 6 ZnCl2 + 3 H2O}
			\item Ten je veden na filtrační papír navlhčený koncentrovaným roztokem \ce{AgNO3}:
			\item \ce{AsH3 + 6 AgNO3 -> Ag3As.3AgNO3 + 3 HNO3}
			\item Vzniklý arsenid následně hydrolyzuje na kovové stříbro:
			\item \ce{Ag3As.AgNO3 + 3 H2O -> 6 Ag + H3AsO3 + 3 HNO3}
		\end{itemize}
	\vfill
}

\frame{
	\frametitle{}
	\vfill
	\begin{columns}
		\begin{column}{.75\textwidth}
			\begin{itemize}
				\item \textit{Oxid arseničný}, \ce{As4O10}, je bílá, hygroskopická a silně toxická látka.
				\item Na vzduchu se rozplývá a je velmi dobře rozpustný ve vodě (230 g ve 100 g vody při 20~$^\circ$C).
				\item Při 300~$^\circ$C dochází k rozkladu za uvolnění kyslíku.
				\item Struktura je tvořena tetraedrickými a oktaedrickými jednotkami, které jsou propojeny přes vrcholy.
				\item Lze jej připravit zahřívání oxidu arsenitého na vzduchu nebo oxidací sulfidových rud.
				\item \ce{As2O3 + O2 <=> As2O5}
				\item \ce{2 As2S3 + 11 O2 -> 2 As2O5 + 6 SO2}
				\item Je anhydridem kyseliny trihydrogenarseničné.
			\end{itemize}
		\end{column}
		\begin{column}{.35\textwidth}
			\begin{figure}
				\adjincludegraphics[width=\textwidth]{img/Arsenic-pentoxide-As-coordination-3D-balls.png}
				\caption*{Koordinační polyedry v \ce{As4O10}.\footnote[frame]{Zdroj: \href{https://commons.wikimedia.org/wiki/File:Arsenic-pentoxide-As-coordination-3D-balls.png}{Ben Mills/Commons}}}
			\end{figure}
		\end{column}
	\end{columns}
	\vfill
}

\frame{
	\frametitle{}
	\vfill
		\begin{figure}
			\adjincludegraphics[height=.65\textheight]{img/Arsenic-pentoxide-3D-balls-A.png}
			\caption*{Krystalová struktura \ce{As4O10}.\footnote[frame]{Zdroj: \href{https://commons.wikimedia.org/wiki/File:Arsenic-pentoxide-3D-balls-A.png}{Ben Mills/Commons}}}
		\end{figure}
	\vfill
}

\frame{
	\frametitle{}
	\vfill
	\begin{itemize}
		\item \textit{Kyselina trihydrogenarsenitá}, \ce{H3AsO3}, je slabá kyselina (p\textit{K}$_{a1}$~=~9,2), která existuje pouze ve vodném roztoku.
		\item Vzniká rozpouštěním oxidu arsenitého ve vodě.
		\item Na rozdíl od kyseliny fosforité má molekula tvar pyramidy ($C_{3v}$).
		\item Tuto strukturu potvrzují i $^1$H NMR spektra.
		\item Tautomer \ce{HAsO(OH)2} nebyl dosud izolován, ani charakterizován.
	\end{itemize}
	\begin{figure}
		\adjincludegraphics[width=.7\textwidth]{img/H3AsO3.png}
	\end{figure}
	\begin{itemize}
		\item \textit{Meyerovou reakcí} vzniká kyselina methylarseničná:\footnote[frame]{\href{https://doi.org/10.1002/cber.188301601316}{Ueber einige anomale Reaktionen}}
		\item \ce{H3AsO3 + CH3I + NaOH -> CH3AsO(OH)2 + NaI + H2O}
	\end{itemize}
	\vfill
}

\frame{
	\frametitle{}
	\vfill
	\begin{itemize}
		\item \textbf{Kyselina trihydrogenarseničná}, \ce{H3AsO4}, je bezbarvá, hygroskopická pevná látka.
		\item Je to trojsytná kyselina:
	\end{itemize}
	\begin{align*}
		\ce{H3AsO4 + H2O &<=> H2AsO$_4^-$ + H3O+ &p\textit{K}$_{a1}$ = 2,19}\\
		\ce{H2AsO$_4^-$ + H2O &<=> HAsO$_4^{2-}$ + H3O+ &p\textit{K}$_{a2}$ = 6,94}\\
		\ce{HAsO$_4^{2-}$ + H2O &<=> AsO$_4^{3-}$ + H3O+ &p\textit{K}$_{a3}$ = 11,5}
	\end{align*}
	\begin{itemize}
		\item Připravuje se oxidací \ce{As2O3} kyselinou dusičnou nebo rozpouštění oxidu arseničného ve vodě.
		\item \ce{As2O3 + HNO3 + 2 H2O -> 2 H3AsO4 + N2O3}
		\item Krystalizací z vodného roztoku získáme hemihydrát.
	\end{itemize}
	\vfill
}

\frame{
	\frametitle{}
	\vfill
	\begin{itemize}
		\item \textit{Oxid antimonitý}, \ce{Sb2O3}, je bílá pevná látka.
		\item Připravuje se spalováním antimonu na vzduchu.
		\item \ce{4 Sb + 3 O2 -> 2 Sb2O3}
		\item Pro výrobu antimonu se získává pražením sulfidu:
		\item \ce{2 Sb2S3 + 9 O2 -> 2 Sb2O3 + 6 SO2}
		\item Je amfoterní, rozpouštěním v alkalických hydroxidech vznikají antimonitany:
		\item \ce{Sb2O3 + 2 NaOH -> 2 NaSbO2 + H2O}
		\item V plynném stavu vytváří molekuly \ce{Sb4O6} s adamantoidní strukturou (podobně jako \ce{P4O6}).
		\item \textit{Oxid antimoničný}, \ce{Sb2O5}, je žlutá pevná látka.
		\item Připravuje se hydrolýzou chloridu antimonitého roztokem amoniaku nebo oxidací antimonu koncentrovanou kyselinou dusičnou.
		\item Antimoničnany jsou tvořeny oktaedry \ce{SbO6}, často propojenými přes vrcholy.
	\end{itemize}
	\vfill
}

\frame{
	\frametitle{}
	\vfill
	\begin{itemize}
		\item Směsný oxid \ce{Sb2O4} je tvořen antimonem v oxidačním čísle III i V.
		\item V přírodě se vyskytuje jako minerál cervantit.\footnote[frame]{\href{https://www.mindat.org/min-936.html}{Cervantite}}
		\item Můžeme ho připravit buď částečnou oxidací oxidu antimonitého vzduchem nebo tepelným rozkladem oxidu antimoničného:
		\item \ce{2 Sb2O3 + O2 -> 2 Sb2O4}
		\item \ce{2 Sb2O5 ->[800 $^\circ$C] 2 Sb2O4 + O2}
	\end{itemize}
	\begin{figure}
		\adjincludegraphics[width=.5\textwidth]{img/Sb2O4_structure.jpg}
		\caption*{Krystalová struktura \ce{Sb2O4}.\footnote[frame]{Zdroj: \href{https://commons.wikimedia.org/wiki/File:Sb2O4_structure.jpg}{Materialscientist/Commons}}}
	\end{figure}
	\vfill
}

\frame{
	\frametitle{}
	\vfill
	\begin{itemize}
		\item \textit{Oxid bismutitý}, \ce{Bi2O3}, je žlutý prášek.
		\item Připravuje se termickým rozkladem bazického dusičnanu nebo hydroxidu bismutitého.
		\item \ce{2 [4 BiNO3(OH)2BiO(OH)] -> 5 Bi2O3 + 8 NO2 + 9 H2O + 2 O2}
		\item Za laboratorní teploty je stabilní monoklinická forma $\alpha$-\ce{Bi2O3} s vrstevnatou strukturou.\footnote[frame]{\href{https://doi.org/10.1002/zaac.19784440118}{On the Structure of Bismuthsesquioxide}}
		\item Při zahřátí nad 717~$^\circ$C přechází na defektní kubickou modifikaci $\delta$-\ce{Bi2O3}.
		\item $\beta$ modifikace má strukturu fluoritu a může obsahovat i \ce{Bi^V}, kdy dochází k obsazení vakancí v krystalové struktuře  anionty \ce{O^{2-}}.
		\item \textit{Oxid bismutičný}, \ce{Bi2O5}, je tmavěčervená pevná látka.
		\item Je nestabilní, snadno uvolňuje kyslík.
		\item Bismutičnany se využívají při stanovení manganu v oceli, kdy mangan oxidují na manganistan. Koncentrace manganistanu se stanovuje kolorimetricky.
	\end{itemize}
	\vfill
}

\subsection{Sulfidy a další chalkogenidy}
\frame{
	\frametitle{}
	\vfill
	\begin{columns}
		\begin{column}{.6\textwidth}
			\begin{itemize}
		\item Sulfidů arsenu známe poměrně velké množství.
		\item \textit{Realgar}, \ce{As4S4} viz minerály arsenu.
		\item \textit{Auripigment}, \ce{As2S3}, je tmavě-žlutá, nerozpustná látka.
		\item Dříve se využíval jako pigment (\textit{královská žluť}).\footnote[frame]{\href{https://ivahonkova.webnode.cz/olejove-barvy/zlute/}{Žlutě}}
		\item Je to V/VI polovodič typu p, vykazuje i fotoresistivní vlastnosti. Díky velikosti zakázaného pásu (2,7 eV) propouští infračervené záření v oblasti 900-16~000~cm$^{-1}$.
		\item Využívá se pro konstrukci optických prvků pro IR spektrometry.
	\end{itemize}
	\end{column}
	\begin{column}{.4\textwidth}
		\begin{figure}
			\adjincludegraphics[width=.95\textwidth]{img/Orpiment_7.jpg}
			\caption*{Auripigment.\footnote[frame]{Zdroj: \href{https://commons.wikimedia.org/wiki/File:Orpiment_7.jpg}{Géry PARENT/Commons}}}
		\end{figure}
	\end{column}
	\end{columns}
	\vfill
}

\frame{
	\frametitle{}
	\vfill
	\begin{columns}
		\begin{column}{.6\textwidth}
			\begin{itemize}
				\item \textit{Sulfid arseničný}, \ce{As2S5}, je oranžová, nerozpustná látka.
				\item Lze ho připravit srážením roztoků arseničných solí sulfanem nebo zahříváním arsenu se sírou.
				\item Ve vařící vodě hydrolyzuje za vzniku kyseliny arsenité:
				\item \ce{As2S5 + 6 H2O ->[100 $^\circ$C] 2 H3AsO3 + 2 S + 3 H2S}
				\item S roztoky alkalických sulfidů poskytuje thioarseničnany:
				\item \ce{2 As2S5 + 6 Na2S -> 4 Na3AsS4}
			\end{itemize}
		\end{column}
		\begin{column}{.4\textwidth}
			\begin{figure}
				\adjincludegraphics[width=\textwidth]{img/As2S5-SEM.jpg}
				\caption*{SEM snímek vulkanického \ce{As2S5}\footnote[frame]{Zdroj: \href{https://commons.wikimedia.org/wiki/File:Color_SEM_2.jpg}{Ppm61/Commons}}}
			\end{figure}
		\end{column}
	\end{columns}
	\vfill
}

\frame{
	\frametitle{}
	\vfill
	\begin{columns}
		\begin{column}{.6\textwidth}
			\begin{itemize}
				\item \textit{Sulfid antimonitý}, \ce{Sb2S3}, je šedá až černá pevná látka.
				\item V přírodě se vyskytuje jako minerál stibnit.
				\item Lze jej připravit zahříváním antimonu se sírou nebo srážením roztoků antimonitých solí sulfanem.
				\item Reakce na mokré cestě je využívána jako gravimetrická metoda stanovení antimonu.
			\end{itemize}
		\end{column}
		\begin{column}{.4\textwidth}
			\begin{figure}
				\adjincludegraphics[width=\textwidth]{img/Sulfid_antimonitý.jpg}
				\caption*{Sulfid antimonitý.\footnote[frame]{Zdroj: \href{https://commons.wikimedia.org/wiki/File:Sulfid_antimonitý.JPG}{Ondřej Mangl/Commons}}}
			\end{figure}
		\end{column}
	\end{columns}
	\ce{2 SbCl3 + 3 H2S ->[95 $^\circ$C] Sb2S3 + 6 HCl}
	\vfill
}

\subsection{Halogenidy \ce{MX3}}
\frame{
	\frametitle{}
	\vfill
	\begin{itemize}
		\item Všechny halogenidy \ce{MX3} jsou stabilní a komerčně dostupné.
		\item Fluoridy lze snadno připravit reakcí oxidu s HF. V případě arsenu nejlépe v bezvodém prostředí:
		\item \ce{M2O3 + 6 HF ->[H2SO4 /CaF2] 2 MF3 + 3 H2O}
		\item Ostatní halogenidy lze připravit reakcí kovu nebo oxidu s halogenem.
	\end{itemize}
	\begin{tabular}{|l|r@{,}l|r@{,}l||l|r@{,}l|r@{,}l|}
		\hline
		\textbf{\ce{MX3}} & \multicolumn{2}{c|}{\textbf{T$_t$ [$^\circ$C]}}
		& \multicolumn{2}{c||}{\textbf{T$_v$ [$^\circ$C]}}
		& \textbf{\ce{MX3}} & \multicolumn{2}{c|}{\textbf{T$_t$ [$^\circ$C]}}
		& \multicolumn{2}{c|}{\textbf{T$_v$ [$^\circ$C]}} \\\hline
		\ce{AsF3} & -8 & 5 & 60 & 4 & \ce{SbBr3} & 96 & 6 & 288 & 0 \\\hline
		\ce{AsBr3} & -12 & 2 & 130 & 2 & \ce{SbI3} & 170 & 5 & 401 & 6 \\\hline
		\ce{AsCl3} & 31 & 1 & 221 & 0 & \ce{BiF3} & 649 & 0 & \multicolumn{2}{c|}{-} \\\hline
		\ce{AsI3} & 146 & 0 & 403 & 0 & \ce{BiCl3} & 227 & 0 & 447 & 0 \\\hline
		\ce{SbF3} & 292 & 0 & 376 & 0 & \ce{BiBr3} & 219 & 0 & 462 & 0 \\\hline
		\ce{SbCl3} & 73 & 4 & 223 & 5 & \ce{BiI3} & 408 & 6 & 542 & 0 \\\hline
	\end{tabular}
	\vfill
}

\subsection{Halogenidy \ce{MX5}}
\frame{
	\frametitle{}
	\vfill
	\begin{itemize}
		\item Známe všechny fluoridy a chlorid antimoničný.
		\item Chlorid arseničný se rozkládá při teplotě $-50\ ^\circ$C. Lze ho připravit oxidací chloridu arsenitého kapalným chlorem při teplotě $-105\ ^\circ$C v přítomnosti UV záření.\footnote[frame]{\href{https://doi.org/10.1002/anie.197603771}{Arsenic Pentachloride, \ce{AsCl5}}}
		\item \ce{AsCl3 + Cl2 ->[UV, -105 $^\circ$C] AsCl5}
	\end{itemize}
	\begin{tabular}{|l|r@{,}l|r@{,}l|}
		\hline
		\textbf{\ce{MX5}} & \multicolumn{2}{c|}{\textbf{T$_t$ [$^\circ$C]}}
		& \multicolumn{2}{c|}{\textbf{T$_v$ [$^\circ$C]}} \\\hline
		\ce{AsF5} & $-$79 & 8 & $-$52 & 8 \\\hline
		\ce{AsCl5} & \multicolumn{4}{c|}{Rozklad při teplotě $-50\ ^\circ$C} \\\hline
		\ce{SbF5} & 8 & 3 & 149 & 5 \\\hline
		\ce{SbCl5} & 2 & 8 & 140 & 0 \\\hline
		\ce{BiF5} & 151 & 4 & 230 & 0 \\\hline
	\end{tabular}
	\vfill
}

\frame{
	\frametitle{}
	\vfill
	\begin{itemize}
		\item \ce{SbCl5} vratně přechází při ochlazení pod $-$55~$^\circ$C na dimerní formu:
	\end{itemize}
	\begin{figure}
		\adjincludegraphics[width=.8\textwidth]{img/SbCl5.png}
	\end{figure}
	\begin{itemize}
		\item Přídavkem chloridu přechází hexachloroantimoničnan:
		\item \ce{SbCl5 + KCl -> KSbCl6}
		\item Reakcí s bezvodým HF vzniká fluorid antimoničný:
		\item \ce{SbCl5 + 5 HF -> SbF5 + 5 HCl}
		\item Reakcí fluoridu antimoničného s fluorem a kyslíkem vzniká hexafluoroantimoničnan dioxygenylu:
		\item \ce{2 SbF5 + F2 + 2 O2 -> 2 [O2]+[SbF6]-}
	\end{itemize}
	\vfill
}
%https://en.wikipedia.org/wiki/Bismuth_vanadate

\frame{
	\frametitle{}
	\vfill
	\begin{itemize}
		\item Kyselina hexafluoroantimoničná, \ce{[H2F][SbF6]}, vzniká reakcí fluoridu antimoničného s fluorovodíkem:
		\item \ce{SbF5 + 2 HF <=> H2F+ + SbF$_6^-$}
		\item Je to nejsilnější známá kyselina, označuje se jako \textit{superkyselina}.
	\end{itemize}
	\begin{figure}
		\adjincludegraphics[width=.6\textwidth]{img/Fluoroantimonic_acid-3D-balls.png}
		\caption*{Model kyseliny hexafluoroantimoničné.\footnote[frame]{Zdroj: \href{https://commons.wikimedia.org/wiki/File:Fluoroantimonic_acid-3D-balls.png}{DFliyerz/Commons}}}
	\end{figure}
	\vfill
}

\frame{
	\frametitle{}
	\vfill
	\begin{itemize}
		\item Anion \ce{SbF$_6^-$}, na rozdíl od \ce{SbCl5}, ochotně vytváří větší částice asociací pomocí fluoridu, např. \ce{[Sb2F11]^-} nebo \ce{[Sb3F16]-}.
	\end{itemize}
	\begin{figure}
		\adjincludegraphics[width=\textwidth]{img/Sb2F11-Sb3F16.png}
	\end{figure}
	\vfill
}

\frame{
	\frametitle{}
	\vfill
	\begin{itemize}
		\item \textit{Magická kyselina} -- vzniká reakcí fluoridu antimoničného s kyselinou fluorsírovou.\footnote[frame]{\href{http://pubsapp.acs.org/subscribe/journals/tcaw/12/i03/pdf/303chronicles.pdf}{A Basic History of Acid-From Aristotle to Arnold}}
		\item Dochází ke koordinaci volného elektronového páru kyslíku do volného orbitalu na antimonu.\footnote[frame]{\href{https://new.societechimiquedefrance.fr/wp-content/uploads/2019/12/2006-301-302-octnov-Olah-p.68.pdf}{Fluorinated superacidic systems}}
		\item Tuto reakci lze dobře monitorovat pomocí $^{19}$F NMR.
		\item Při nižší koncentraci \ce{SbF5} vzniká monoadukt, při vyšší začíná vznikat adukt 1:2.
		\item \ce{SbF5 + 2 HSO3F -> [SbF5.SO3F]- + H2SO3F+}
		\item \ce{2 SbF5 + 2 HSO3F -> [2SbF5.SO3F]- + H2SO3F+}
		\item Kyselost tohoto systému lze zvýšit přídavkem tří molů \ce{SO3}.
		\item Roztoky \ce{SbF_{5}/HSO3F/SO3} se označují jako superkyselá média.
	\end{itemize}
	\vfill
}

\frame{
	\frametitle{}
	\vfill
	\begin{figure}
		\adjincludegraphics[width=.9\textwidth]{img/Magic_acid_structure.png}
	\end{figure}
	\vfill
}

\frame{
	\frametitle{}
	\vfill
	\begin{itemize}
		\item \textit{Hammettova kyselostní funkce}  ($H_0$) -- používá se k vyjádření kyselosti koncentrovaných roztoků silných kyselin a superkyselin.
		\item Lze ji zjistit pomocí interakce kyseliny s velmi slabou bazí:
		\item $H_0 = pK_{BH^+} + \log \frac{[B]}{[BH^+]} = -\log(a_{H^+} \frac{\gamma_B}{\gamma_{BH^+}})$
	\end{itemize}
	\begin{tabular}{|l|l|}
		\hline
		Kyselina sírová & $-$12,0 \\\hline
		Kyselina chloristá & $-$12,78 \\\hline
		Kyselina trifluormethansulfonová & $-$14,1 \\\hline
		Fluorovodík & $-$15,1 \\\hline
		Kyselina fluorsírová & $-$15,1 \\\hline
		Magická kyselina & $-$19,2 \\\hline
		Kyselina hexafluorantimoničná & $-$21 -- $-$23 \\\hline
	\end{tabular}
	\vfill
}

\frame{
	\frametitle{}
	\vfill
	\begin{itemize}
		\item \textbf{Fluorid bismutičný}, \ce{BiF5}, je jedna z mála sloučenin bismutu v oxidačním stavu V.
		\item Je to velmi silné fluorační a oxidační činidlo.
		\item Lze ho připravit fluorací \ce{BiF3} pomocí fluoru nebo fluoridu chloritého:\footnote[frame]{\href{https://doi.org/10.1002/zaac.19895760128}{Synthesis and properties of pentavalent antimony and bismuth fluorides}}
		\item \ce{BiF3 + F2 ->[500 $^\circ$C] BiF5}
		\item \ce{BiF3 + ClF3 ->[350 $^\circ$C] BiF5 + ClF}
		\item Dokáže velmi účinně fluorovat jiné sloučeniny:\footnote[frame]{\href{https://doi.org/10.1021/ja01533a009}{Preparation, Properties and Reactions of Bismuth Pentafluoride}}
		\item \ce{BiF5 + UF4 ->[150 $^\circ$C] BiF3 + UF6}
		\item \ce{3 BiF5 + Br2 ->[180 $^\circ$C] 3 BiF3 + 2 BrF3}
		\item \ce{3 BiF5 + S ->[25 $^\circ$C] 3 BiF3 + SF6}
		\item Krystalová struktura je tvořena řetězci oktaedrů \ce{BiF6} spojenými vrcholy v poloze trans.\footnote[frame]{\href{https://doi.org/10.1002/zaac.19713840204}{Zur Kristallstruktur von Wismutpentafluorid}}
	\end{itemize}
	\vfill
}

\frame{
	\frametitle{}
	\vfill
	\begin{columns}
		\begin{column}{.4\textwidth}
			\begin{figure}
				\adjincludegraphics[width=\textwidth,rotate=90]{img/Bismuth-pentafluoride-chain.png}
				\caption*{Řetězec oktaedrů \ce{BiF6}.\footnote[frame]{\href{https://en.wikipedia.org/wiki/File:Bismuth-pentafluoride-chain-from-xtal-1971-3D-balls.png}{Zdroj: Ben Mills/Commons}}}
			\end{figure}
		\end{column}

	\begin{column}{.6\textwidth}
			\begin{figure}
			\adjincludegraphics[width=.8\textwidth]{img/Bismuth-pentafluoride-chain-packing.png}
			\caption*{Uspořádání řetězců v krystalu \ce{BiF5}.\footnote[frame]{\href{https://en.wikipedia.org/wiki/File:Bismuth-pentafluoride-chain-packing-from-xtal-1971-3D-balls.png}{Zdroj: Ben Mills/Commons}}}
		\end{figure}
	\end{column}
	\end{columns}
	\vfill
}

\subsection{Další halogenidy}
\frame{
	\frametitle{}
	\vfill
	\begin{itemize}
		\item Tendence ke tvorbě směsných halogenidů je nižší než u fosforu.
		\item Výměna halogenidů mezi \ce{AsF3} a \ce{AsCl3} (za laboratorní teploty) byla prokázána pomocí NMR a Ramanovy spektroskopie.
		\item U pentahalogenidů je tvorba směsných sloučenin běžnější.
		\item \ce{2 AsCl3 + 2 AsF3 + Cl2 -> 2 [AsCl4][AsF6]}
		\item \ce{AsCl3 + SbCl5 + Cl2 -> [AsCl4][SbCl6]}
	\end{itemize}
	\vfill
}

\subsection{Vanadičnan bismutitý}
\frame{
	\frametitle{}
	\vfill
	\begin{columns}
		\begin{column}{.65\textwidth}
			\begin{itemize}
				\item Vanadičnan bismutitý, \ce{BiVO4}.
				\item Světle žlutý prášek, je studován jako fotokatalyzátor pro rozklad vody viditelným světlem.\footnote[frame]{\href{https://doi.org/10.1039/C4EE03271C}{Visible-light driven heterojunction photocatalysts for water splitting – a critical review}}
				\item Využívá se jako žlutý pigment, připravuje se srážením nebo reakcí \ce{Bi2O3} a \ce{V2O5} v pevné fázi.
				\item V přírodě se vyskytuje ve formě vzácných minerálů: dreyeritu,\footnote[frame]{\href{https://www.mindat.org/min-1320.html}{Dreyerite}} clinbisvanitu\footnote[frame]{\href{https://www.mindat.org/min-1067.html}{Clinobisvanite}} a pucheritu\footnote[frame]{\href{https://www.mindat.org/min-3306.html}{Pucherite}}.
			\end{itemize}
		\end{column}

		\begin{column}{.4\textwidth}
			\begin{figure}
				\adjincludegraphics[height=.3\textheight]{img/Bismuthvanadat.jpg}
				\caption*{Vanadičnan bismutitý.\footnote[frame]{Zdroj: \href{https://commons.wikimedia.org/wiki/File:Bismuthvanadat.jpg}{FK1954/Commons}}}
			\end{figure}
		\end{column}
	\end{columns}
	\vfill
}

\subsection{Organokovové sloučeniny}
\frame{
	\frametitle{}
	\vfill
	\begin{itemize}
		\item Všechny tři prvky vytváří organokovové sloučeniny v oxidačních číslech III a V.
		\item Stabilita klesá se stoupající hmotností prvku.
		\item Sloučeniny antimonu a bismutu nejsou tak široce prostudovány jako sloučeniny arsenu.
		\item První organokovovou sloučeninou arsenu byl kakodyl, neboli tetramethyldiarsan.
		\item Arsen vytváří arsonové a arsinové kyseliny (podobně jako fosfor).
	\end{itemize}
	\begin{figure}
		\adjincludegraphics[width=\textwidth]{img/As-acids.png}
	\end{figure}
	\vfill
}

\frame{
	\frametitle{}
	\vfill
	\begin{figure}
		\adjincludegraphics[width=\textwidth]{img/As-aromatic.png}
	\end{figure}
	\begin{itemize}
		\item Známe řadu aromatických sloučenin, ve kterých má arsen oxidační číslo III a~koordinační číslo~2.
		\item \textit{Arsabenzen} je bezbarvý plyn, odolný vůči hydrolýze slabými kyselinami a zásadami.
		\item Připravuje se z 1,4-pentadiynu:\footnote[frame]{\href{https://doi.org/10.1271/kagakutoseibutsu1962.33.100}{Chemistry of heterobenzenes containing group 15 element}}
	\end{itemize}
	\begin{figure}
		\adjincludegraphics[width=\textwidth]{img/Arsabenzen.png}
	\end{figure}
	\vfill
}

\frame{
	\frametitle{}
	\vfill
	\begin{itemize}
		\item Molekula arsabenzenu je planární.
		\item \textit{Stibabenzen} se připravuje podobně jako arsabenzen.\footnote[frame]{\href{https://doi.org/10.1021/ja00753a069}{Stibabenzene}}
		\item Posledním členem řady je \textit{bismabenzen}, ten je ale nestabilní a dosud nebyl izolován v čistém stavu.
		\item Byly připraveny jeho deriváty, které jsou stabilnější.\footnote[frame]{\href{https://www.chemistryworld.com/news/chemists-create-stable-bismuth-benzene-derivative/1017447.article}{Chemists create stable bismuth benzene derivative}}
	\end{itemize}
	\begin{figure}
		\adjincludegraphics[width=\textwidth]{img/Bond_lengths_of_group_15_heterobenzenes_and_benzene.png}
		\caption*{Délky vazeb a vazebné úhly benzenu a heterobenzenech.\footnote[frame]{\href{https://commons.wikimedia.org/wiki/File:Bond_lengths_of_group_15_heterobenzenes_and_benzene.svg}{Zdroj: Sponk/Commons}}}
	\end{figure}
	\vfill
}

\frame{
	\frametitle{}
	\vfill
	\begin{figure}
		\adjincludegraphics[width=\textwidth]{img/BiC5H5-Si2.png}
		\caption*{Stabilní derivát bismabenzenu\footnote[frame]{\href{https://www.chemistryworld.com/news/chemists-create-stable-bismuth-benzene-derivative/1017447.article}{Chemists create stable bismuth benzene derivative}}}
	\end{figure}
	\vfill
}

\section{Biologie}
\subsection{Arsen}
\frame{
	\frametitle{}
	\vfill
	\begin{itemize}
		\item Arsenité sloučeniny jsou toxičtější než arseničné.
		\item Atoxyl byl využíván při léčbě spavé nemoci.
		\item Některé organické sloučeniny arsenu byly dříve využívány při léčbě syfilidy.
		\item Sloučeniny arsenu se využívají při léčbě africké trypanosomiasy.\footnote[frame]{\href{https://doi.org/10.1186/1756-3305-3-15}{The development of drugs for treatment of sleeping sickness: a historical review}}
	\end{itemize}
	\begin{figure}
		\adjincludegraphics[width=.8\textwidth]{img/Atoxyl.png}
	\end{figure}
	\vfill
}

\subsection{Antimon}
\frame{
	\frametitle{}
	\vfill
	\begin{itemize}
		\item Kovový antimon neovlivňuje lidské zdraví.
		\item Oxid antimonitý a další nerozpustné antimonité sloučeniny jsou nebezpečné při vdechování.
		\item Otrava antimonitými sloučeninami je podobná otravě arsenikem.
		\item Oxid antimonitý je také potenciálně karcinogenní.\footnote[frame]{\href{https://doi.org/10.1016/j.mrfmmm.2003.07.012}{Cobalt and antimony: genotoxicity and carcinogenicity}}
	\end{itemize}
	\vfill
}

\input{../Last}

\end{document}